\chapter{Introduction} \label{chap1}
Code review is probably the single-most effective technique for identifying security flaws. When used together with automated tools and manual penetration testing, code review can significantly increase the cost effectiveness of an application security verification effort.
Manual security code review provides insight into the "real risk" associated with insecure code. This is the single most important value from a manual approach. A human reviewer can understand the context for certain coding practices, and make a serious risk estimate that accounts for both the likelihood of attack and the business impact of a breach.

\section{Purpose}
Code reviews have two purposes. Their first purpose is to make sure that the code that is being produced has sufficient quality to be released. Code reviews are very effective at finding errors of all types, including those caused by poor structure, those that don't match business process, and also those simple omissions.
The second purpose is as a teaching tool to help developers learn when and how to apply techniques to improve code quality, consistency, and maintainability. 

\section{Coding Conventions}
Coding conventions are a set of guidelines for a specific programming language that recommend programming style, practices and methods for each aspect of a piece program written in this language. These conventions usually cover file organization, indentation, comments, declarations, statements, white space, naming conventions, programming practices, programming principles, programming rules of thumb, architectural best practices, etc. These are guidelines for software structural quality. Coding conventions are only applicable to the human maintainers and peer reviewers of a software project.
We decided to use \textit{"Oracle Java Code Conventions"}, for our code inspection.

\section{Issues Checklist}
This is a comprehensive tabular checklist of possible issues that we used during the inspection process.

\input{cpt/Checklist.tex}

\section{References}
\begin{itemize}
	\item Assignment 3 - Code Inspection.pdf.
	\item Link referenced to \textit{"Oracle Java Code Conventions"}:\\
	\href{http://www.oracle.com/technetwork/java/codeconvtoc-136057.html} {http://www.oracle.com/technetwork/java/codeconvtoc-136057.html}
\end{itemize}

\section{Document Overview}
This document is essentially structured in three parts:
\begin{itemize}
	\item \textbf{Section \ref{chap1}} $\rightarrow$ Introduction: it gives a description of the document and gives general information about "Coding Conventions" and the "Checklist" used in order to detect issues in the code.
	\item \textbf{Section \ref{chap2}} $\rightarrow$ Code Inspection Process : it gives a description of the classes assigned and contains all the problems detected during the code inspection.
	\item \textbf{Section \ref{chap3}} $\rightarrow$ Other Information: this part contains information about the total working hours of each group member.
\end{itemize}