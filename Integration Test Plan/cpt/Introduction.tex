\chapter{Introduction} \label{chap1}
Integration testing confirms that each piece of the application interacts as designed and that all functionality is working. Integration testing includes interactions between all layers of an application, including interfaces to other applications, as a complete end-to-end test of the functionality.
The development team will be responsible for the creation of the integration test scripts in accordance to the integration test plan.  A developer will be chosen by the team who will be responsible for execution of the test scripts and certifying that the integration testing is complete.

\section{Revision History}
\textit{Version 1.0}, date 21/01/2016

\section{Purpose}
The purpose of the integration test plan is to describe the necessary tests to verify that all of the components of MyTaxiService are properly assembled.  Integration testing ensures that the unit-tested modules interact correctly.

\section{Scope}
This document refers to the developing of an application called MyTaxiService, which is aimed to improve the quality and the efficiency of the taxi service of a large city by using localization, smartphones and IT technologies.

\section{List of Definitions and Abbreviations}
\begin{description}
	\item \underline{RASD}: Requirement Analysis and Specification Document
	\item \underline{ITPD}: Integration Test Plan Document
	\item \underline{IT}: Integration Test
	\item \underline{TP}: Test Procedure 
\end{description}

\section{List of Reference Documents}
List of all reference documents:
\begin{itemize}
	\item Project Description: Assignments 1 and 2 (RASD and DD).pdf
	\item RASD: RASD v2.0-CrippaGalluzziLattarulo.pdf
	\item Design Document: DesignDocument v1.0-CrippaGalluzziLattarulo.pdf
	\item ITPD: Assignment 4 - integration test plan.pdf
\end{itemize}

\section{Document Overview}
This document is essentially structured in five parts:
\begin{itemize}
	\item \textbf{Section \ref{chap1}} $\rightarrow$ Introduction: defines the purpose, the scope and an overview of this document.
	\item \textbf{Section \ref{chap2}} $\rightarrow$ Integration Strategy: defines all the items to be tested and the integration testing approach.
	\item \textbf{Section \ref{chap3}} $\rightarrow$ Individual Steps and Test Description: describes the type of test that will be used to verify that the elements perform as expected.
	\item \textbf{Section \ref{chap4}} $\rightarrow$ Tools and Test Equipment Required: defines all tools and test equipment needed to accomplish the integration.
	\item \textbf{Section \ref{chap5}} $\rightarrow$ Program Stubs and Test Data Required: defines any program stubs or special test data required.
\end{itemize}