\chapter{Algorithm Design} \label{chap3}
\subsubsection{Queue Management:}
It?s possible to create a queue implementing the Java Interface Queue, which is a collection designed for holdings elements prior to processing. Besides basic Collection operations, queues provide additional insertion, extraction, and inspection operations. Queues typically order elements in a FIFO (first-in-first-out) manner. The \textit{head} of the queue is that element which would be removed by a call to \textbf{remove()} or \textbf{poll()}. In a FIFO queue, all new elements are inserted at the \textit{tail} of the queue.
This Interface provides some methods to manage the queue:
\begin{itemize}
	\item \textbf{add(e)} $\rightarrow$ Method that allow to insert a new element e in the queue
	\item \textbf{remove()} and \textbf{poll()} $\rightarrow$ Methods that remove and return the head of the queue. Methods differ only in their behavior when the queue is empty: the remove() method throws an exception, while the poll() method returns null.
	\item \textbf{element()} and \textbf{peek()} $\rightarrow$ Methods that return, but do not remove, the head of the queue.
\end{itemize}

\subsubsection{Research of a Taxi Driver in the queue:}
The system uses a FIFO queue to manage the requests to the taxi drivers. It will use the method element() in order to extract without removing the first taxi driver of the queue and it will send the request to him. If the first taxi driver accepts the request, this will be assigned to him and he will be dequeued, otherwise he will be dequeued and the system will use the new first element of the queue to detect the new taxi driver who will receive the request, and so on until a taxi driver accepts the request.

\begin{algorithm}[H]
\caption{Research of a Taxi Driver}
\begin{algorithmic}[1]
\Procedure{SearchTaxiDriver}{$Q$}
\If{$(Q.head == Q.tail)$}\Comment{Queue is empty}
	\Else
	\For{$i:= 0$ \textbf{to} $Q.lenght$}
		\State $x \gets Dequeue(Q)$\Comment{Classical operation for managing the extraction of an element from the Queue}
		\State $acceptance \gets Call(x)$\Comment{Function that represent the call that the system makes to the driver in order to propose a ride. It \textbf{returns true} if the taxi driver accepts the call and is willing to make the ride, otherwise \textbf{returns false} if the taxi driver declines the call}
		\If{$(acceptance == true)$}
			\State $\textbf{return} \: x$
		\Else
			\State $Enqueue(Q, x)$\Comment{Classical operation for managing the insertion of an element from the Queue}
		\EndIf
	\EndFor
\EndIf
\EndProcedure
\end{algorithmic}
\end{algorithm}

\subsubsection{Shared Ride Management:}
This is the algorithm that is responsible to manage the Shared Rides. Here there are some general information about the algorithm: \\
There is one sharing list per every taxi area ordered by the starting time of the ride. Match between elements of the list is done under a time window of 10 minutes after the selected starting time (for reservation) and after the moment of the call (for request).When is time to start assigning a driver to the request/reservation and start the ride, the system deletes the corresponding element from the list (is always the head of the list because of the ordering). There is also a buffer (Figure \ref{fig:buff}) to make sequential the adding of new requests/reservations to the list and avoid conflicts.\\
The flowchart diagram in Figure \ref{fig:genalg} represent the generic algorithm which describes how the system manage the functionality of "Ride Sharing".

\begin{figure}[htbp]
\centering
\includegraphics[width=\textwidth]{cpt/img/BufferQueue}
\caption{Representation of interaction between Queue and Buffer}
\label{fig:buff}
\end{figure}
\clearpage

\begin{figure}[htbp]
\centering
\includegraphics[width=\textwidth]{cpt/img/GenericAlgorithm}
\caption{Representation of the Shared Ride Management Algorithm}
\label{fig:genalg}
\end{figure}
\clearpage

The following algorithms describe in details the behavior of the list and how the Shared Ride Management Algorithm works:

\begin{algorithm}[H]
\caption{Check for Compatibility}
\begin{algorithmic}[1]
\Procedure{CheckForCompatibility}{$x, List$}
\For{$each \: element \: e \: \textbf{in} \: Sublist(x)$}\Comment{\textit{Sublist(x)} indicates the portion of the list starting from the successor of element \textit{x}}
	\If{$((e.startTime \leq x.startTime + 10)$ \textbf{\&\&} $(e.destinationArea == x.DestinationArea)$ \textbf{\&\&} $(e.flagAsFullRide == false))$}
		\State $\textit{MergeRides(x, e)}$
		\State ${\textbf{break()}}$
	\EndIf
\EndFor
\EndProcedure
\end{algorithmic}
\end{algorithm}

\begin{algorithm}[H]
\caption{Merge Rides}
\begin{algorithmic}[1]
\Procedure{MergeRides}{$x, y$}
\State \textit{Creates a new element z with the same originArea and destinationArea of x and y}
\If{$x.startTime < y.startTime$}
	\State $z.startTime \gets x.startTime$
	\Else
	\State $z.startTime \gets y.startTime$
\EndIf
\State \textit{All the other information about the request/reservation are copied from x and y into z}
\EndProcedure
\end{algorithmic}
\end{algorithm}

\begin{algorithm}[H]
\caption{Insert an element in List keeping the order}
\begin{algorithmic}[1]
\Procedure{InsertKeepengOrder}{$x, List$}
\For{$each \: element \: e \: \textbf{in} \: List$}
	\If{$(e.startTime \geq x.startTime)$}
		\State $e.prev.next \gets x$\Comment{\textit{e.prev.next} indicates the attribute next of the element pointed by \textit{e.prev}}
		\State $x.prev. \gets e.prev$
		\State $x.next \gets e$
		\State $e.prev \gets x$
		\State ${\textbf{break()}}$
	\EndIf
\EndFor
\EndProcedure
\end{algorithmic}
\end{algorithm}