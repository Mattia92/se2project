\chapter{Introduction} \label{chap1}

\section{Purpose}
This document represent the \textit{Requirement Analysis and Specification Document} (RASD). The main goal of this document is to completely describe the system in terms of functional and non functional requirements, analyze the real need of the customer to modeling the system, show the constraints and the limits of the software and simulate the typical use cases that will occur after the development. This document is intended to all developers and programmers who have to implement the requirements, to system analysts who want to integrate other systems with this one, and could be used as a contractual basis between the customer and the developer.

\section{Scope}
The aim of the project is to optimize the taxi service of a city using modern techniques and technologies to better manage the distribution and the availability of the taxicabs, to shorten the waiting time for passengers and to reduce costs for passengers and traffic in the entire city using taxi sharing. 
The system is implemented as a mobile application associated to an external, high precision GPS receiver for the taxi drivers and as a mobile application or a web interface for the passengers.

\section{Domain Properties}
\noindent We suppose that the following properties hold in the analyzed domain:
\begin{itemize}
	\item [\textbf{D01}] : Accurate taxicabs positions are known by the GPS
	\item [\textbf{D02}] : Taxi drivers correctly signal their availability
	\item [\textbf{D03}] : If a passenger requests or reserves a taxi, he will then take the ride from the specified origin to the specified destination
	\item [\textbf{D04}] : A passenger doesn't change the origin or the destination of a ride after the reservation
	\item [\textbf{D05}] : If a taxi driver confirms a ride, then he will reach the passenger location on time according to the reservation hour or the waiting time calculated by the system
	\item [\textbf{D06}] : If a taxi driver confirms a ride, he will complete it
\end{itemize}

\section{Proposed System}
The system will be composed by a server running all the business logic, generating dynamic web pages and managing all the accesses to the data sources, and by a number of clients implemented as mobile application deployed on Android or a web application using the JEE platform. GPS raw data will be provided by a specific GPS receiver, installed in every taxicab and provided by a specialized company.

\section{Goals}
\noindent \textbf{Visitors} should be able to:
\begin{itemize}
	\item [\textbf{G01}] : Sign up 
	\item [\textbf{G02}] : Log in\\
\end{itemize}

\noindent \textbf{Passengers} should be able to:
\begin{itemize}
	\item [\textbf{G03}] : Request a taxi
	\item [\textbf{G04}] : Reserve a taxi
	\item [\textbf{G05}] : Receive information about the incoming taxi (confirmation, taxi ID, waiting time)
	\item [\textbf{G06}] : Share the ride
	\item [\textbf{G07}] : Receive receipts after each completed ride\\
\end{itemize}

\noindent \textbf{Taxi drivers} should be able to:
\begin{itemize}
	\item [\textbf{G08}] : Signal whether they are available or not
	\item [\textbf{G09}] : Receive incoming requests
	\item [\textbf{G10}] : Accept or decline incoming requests
	\item [\textbf{G11}] : Visualize information about the optimal route\\
\end{itemize}

\noindent \textbf{Developers} should be able to:
\begin{itemize}
	\item [\textbf{G12}] : Add new features
\end{itemize}

\section{Assumptions} \label{assump}
\begin{itemize}
	\item [\textbf{A01}]: Given the collected data about the distribution and the total number of the taxicabs, we assume that the coverage of the taxicabs among the zones is almost uniform; therefore the probability of having an empty queue is reasonably low.
	\item [\textbf{A02}]: Given the collected data about the acceptance rate of the taxi drivers, we assume that the probability of reaching the end of the queue without any acceptance is reasonably low.
	\item [\textbf{A03}]: We assume that the maximum number of people sharing the same ride is 3.
	\item [\textbf{A04}]: We assume that the taxi driver will correctly follow the optimal route suggested by the system.
\end{itemize}

\section{Stakeholders Identification}
\begin{itemize}
	\item The government of the city in which the system will be used
	\item The citizens of the city in which the system will be used
	\item The taxi drivers of the city in which the system will be used
	\item A specific IT company which provides all the technology (GPS receivers, smartphones for taxi drivers, mainframes and computational power) in exchange of raw GPS data provided by the receiver used for data mining purposes
\end{itemize}

\section{Definitions, acronyms and abbreviations}
\begin{description}
	\item \underline{PASSENGER}: is a citizen that benefits of the taxi service
	\item \underline{TAXI DRIVER}: is the owner of a taxicab, and provides a taxi service to citizens
	\item \underline{RIDE}: is the act of make use of the taxi service (provided by a taxi driver) by a passenger
	\item \underline{REQUEST}: is the act performed by a passenger when he immediately needs a taxicab in the location in which he is 
	\item \underline{RESERVATION}: is the act performed by a passenger when he needs a taxicab in a certain future time (at maximum in 2 hours from now) at a certain specific location
	\item \underline{ROUTE}: is the path through the city from the origin of the specific ride to the destination of the specific ride
\end{description}

\section{References}
\begin{itemize}
	\item Specification Document: MyTaxiService Project AA 2015-2016.pdf.
	\item IEEE Std 830-1998 IEEE Recommended Practice for Software Requirements Specifications.
	\item IEEE Std 1016tm-2009 Standard for Information Technology - System Design - Software Design Descriptions.
\end{itemize}

\section{Overview}
This document is essentially structured in eight parts:
\begin{itemize}
	\item \textbf{Section \ref{chap1}} $\rightarrow$ Introduction: it gives a description of document and gives general information about the software product with more focus about constraints and assumptions.
	\item \textbf{Section \ref{chap2}} $\rightarrow$ Actors Identification: it gives a description of all actors of the system.
	\item \textbf{Section \ref{chap3}} $\rightarrow$ Specific Requirements: this part lists all the functional and non functional requirements that are part of the system.
	\item \textbf{Section \ref{chap4}} $\rightarrow$ Scenarios Identification: it gives a description of typical scenarios.
	\item \textbf{Section \ref{chap5}} $\rightarrow$ UML Models: to give an easy way to understand all functionality of this software, this section is filled with UML diagrams and it contains also some use cases.
	\item \textbf{Section \ref{chap6}} $\rightarrow$ Alloy Modeling: this part contains the code used for the analysis of consistency of the Class Diagram and words generated by Alloy Analyzer in order to understand that the model is consistent.
	\item \textbf{Section \ref{chap7}} $\rightarrow$ Used Tools: this part contains some information about the software used to realize this document.
\end{itemize}